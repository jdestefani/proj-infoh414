% DO NOT COMPILE THIS FILE DIRECTLY!
% This is included by the other .tex files.

\section*{Outline}
\frame{\tableofcontents}

\section{Context of the research}
\begin{frame}[t,fragile]{Research context}
The evaluation of interactive segmentation is generally done by means of user experimentation.\newline
This method is effective, but also labor-intensive and time-consuming.\newline 
The paper proposes an automated approach imitating human behaviour, evaluating it using 4 different algorithms:
\begin{enumerate}
\item \textbf{BPT: }Interactive segmentation using Binary Partition Trees
\item \textbf{IGC: } Interactive Graph Cuts
\item \textbf{SRG: } Seeded Region Growing 
\item \textbf{SIOX: } Simple Interactive Object Extraction
\end{enumerate}
These algorithms provide a good coverage of the underlying algorithmic approaches described in the literature for
object extraction from natural scenes.

\end{frame}

\section{Theoretical notions}

\begin{frame}[t,fragile]{User-based segmentation}
\begin{center}
\includegraphics[height=.8\textheight,keepaspectratio]{AutomationSchema}
\end{center}
\end{frame}

\begin{frame}[t,fragile]{Strategies overview}
\begin{itemize}
\item \textbf{Strategy 1 -} Deterministic choice of ground truth objects centers as seed points.
\item \textbf{Strategy 2 -} Non-deterministic choice of seeds points with distance-proportional probability.
\item \textbf{Strategy 3 -} Computation of the shortest seed line defining an acceptable segmentation.
\item \textbf{Strategy 4 -} Computation of the shortest seed line defining an acceptable segmentation with preference for those passing near the center of the object.
\end{itemize}
\end{frame}

\begin{frame}[t,fragile]{Automated Strategy 1 - Overview}
Seed points are chosen in a way that :
\begin{itemize}
  \item \textbf{(Initialization) -}Intuitively, points closer to the center of the ground truth object are marked as object point, while points clearly outside of it are marked as background
  \item \textbf{(Update) -} Update seed points are chosen within large misclassified areas
\end{itemize}
\textbf{Strong points :}
\begin{itemize}
  \item Efficient computation using fast 2D Euclidian distance. 
  \item Determinism.
  \item Useful baseline for comparisons.
\end{itemize}
\textbf{Weak points :}
\begin{itemize}
  \item Determinism does not allow to test the robustness of the strategy.
  \item Seeds have the form of pixel blobs instead of curves.
\end{itemize}
\end{frame}


\begin{frame}[t,fragile]{Automated Strategy 2 - Overview}
The strategy is similar to the previous one, the main difference being a non deterministic choice of the points according to a probability distribution which is proportional
to the distance of points within the candidate set to points outside this set. \newline
\textbf{Strong points :}
\begin{itemize}
  \item Efficient computation using inversion method to compute the probability.
  \item Non-Determinism allows to test repeatability, hence robustness.
\end{itemize}
\textbf{Weak points :}
\begin{itemize}
  \item Seeds have the form of pixel blobs instead of curves.
\end{itemize}
\end{frame}

\begin{frame}[t,fragile]{Automated Strategy 1-2 - Example}
\begin{center}
\includegraphics[height=.8\textheight,keepaspectratio]{Strategy1Ex}
\end{center}
\end{frame}


\begin{frame}[t,fragile]{Automated Strategy 3 - Overview}
The seed set will have the form of an automatically generated line, obtained by applying Dijkstra's shortest path algorithm on the adjacency graph of the candidate pixels.
The line will be eventually expanded using a brush function to ensure a better coverage of the object.\newline
\textbf{Strong points :}
\begin{itemize}
  \item Efficient computation of connection relationship and shortest path using Dijkstra's algorithm.
  \item Seed line define a better coverage of the ground truth object than pixel blobs.
\end{itemize}
\textbf{Weak points :}
\begin{itemize}
  \item Shortest seed line tends to be closer to the border than desired.
\end{itemize}
\end{frame}

\begin{frame}[t,fragile]{Automated Strategy 3 - Example}
\begin{center}
\includegraphics[height=.5\textheight,keepaspectratio]{Strategy3Ex}
\end{center}
\end{frame}

\begin{frame}[t,fragile]{Automated Strategy 4 - Overview}
This strategy is identical to the previous one, except for the weight assigned to each edge in the adjacency graph, which is modified in order to produce
seed lines which pass closer to the center of the ground truth object.\newline
\textbf{Strong points :}
\begin{itemize}
  \item Efficient computation of connection relationship and shortest path using Dijkstra's algorithm.
  \item Seed line define a better coverage of the ground truth object than pixel blobs.
\end{itemize}
\textbf{Weak points :}
\begin{itemize}
  \item Shortest seed line tends to be closer to the border than desired.
\end{itemize}
\end{frame}

\begin{frame}[t,fragile]{Automated Strategy 3 - Example 1}
\begin{center}
\includegraphics[height=.5\textheight,keepaspectratio]{Strategy3Ex2}
\end{center}
\end{frame}

\begin{frame}[t,fragile]{Automated Strategy 3 - Example 2}
\begin{center}
\includegraphics[height=.7\textheight,keepaspectratio]{Strategy4Ex}
\end{center}
\end{frame}

\section{Results Analysis}
\begin{frame}[t,fragile]{Evaluation method}
Given an input image and the corresponding ground truth:
\begin{enumerate}
\item Select strategy and algorithm.
\item Process image according to strategy and algorithm.
\item Compute object accuracy and border accuracy.
\item Update seeds.
\item If maximum step number has been reached stop, else goto 2.
\end{enumerate}
\begin{itemize}
  \item Non deterministic strategy are rerunned 5 times in order to evaluate repeatability.
  \item The maximum number of steps is equal to 100, which is unrealistic compared to human interaction, to allow stabilization of the results and to observe the strategy behavior over a long run.
  \item Object and border accuracy will result in a time series of values.
  \item Accuracy =  $\frac{TruePositive}{TruePositive+FalsePositive+FalseNegative}$ (Jaccard Index)
\end{itemize}
\end{frame}

\begin{frame}[t,fragile]{Evaluation metrics}
\begin{itemize}
  \item \textbf{(Profile) - Time-Accuracy Profile } Average accuracy profile across time.
  \item \textbf{(Aggregate) - Final Accuracy } Accuracy achievable across a reasonable amount of time.
  \item \textbf{(Aggregate) - Integrated Accuracy } Accuracy integrated over time series data expanded in order to be comparable 
\end{itemize}
\begin{itemize}
  \item These metrics can be averaged across all the object in the dataset to assess overall system performance.
  \item Integrated accuracy is computed as a discrete summation of the area below the expanded time series.
  \item Useful because it can be easily normalized with respect to the unity rectangle and briefly express the trend of average accuracy.
  \item  Must be computed also for user interaction where steps are not unit spaced (i.e resampling or trapezoid rule).
\end{itemize}
\end{frame}

\begin{frame}[t,fragile]{User Interaction Results}
\begin{center}
\includegraphics[height=.8\textheight,keepaspectratio]{MeanAccuracyUser}
\end{center}
\end{frame}

\begin{frame}[t,fragile]{Automation Results - Accuracy Profile}
\begin{center}
\includegraphics[height=.8\textheight,keepaspectratio]{MeanAccuracyAutomation}
\end{center}
\end{frame}

\begin{frame}[t,fragile]{Automation Results - Final Accuracy}
\begin{center}
\includegraphics[height=.7\textheight,keepaspectratio]{MeanAccuracyFinal}
\end{center}
\end{frame}

\begin{frame}[t,fragile]{Automation Results - Integrated Accuracy}
\begin{center}
\includegraphics[height=.7\textheight,keepaspectratio]{MeanAccuracyIntegrated}
\end{center}
\end{frame}

\begin{frame}[t,fragile]{Correlation metrics}
\begin{itemize}
  \item \textbf{Pearson's product-moment coefficient} 
  \item \textbf{Spearman's $\rho$ rank coefficient}
\end{itemize}
\begin{itemize}
  \item User (time-based) and automated (step-based) data must be aligned to perform correlation analysis.
  \item Step-accuracy correlation analysis is limited to 60 step, because of results stabilization and limited number of user interactions.
  \item Aggregated measures are averaged across different users (for human interaction) or different runs (for non-deterministic strategies) before performing correlation analysis.
  \item High correlation between user and automated accuracy profile will show that the proposed method is well-emulating human behavior. 
\end{itemize}
\end{frame}

\begin{frame}[t,fragile]{Automation Results - Step-accuracy correlation}
\begin{center}
\includegraphics[height=.6\textheight,keepaspectratio]{StepwiseCorrelation}
\end{center}
\end{frame}

\begin{frame}[t,fragile]{Automation Results - Aggregate values correlation}
\begin{center}
\includegraphics[height=.7\textheight,keepaspectratio]{AggregateCorrelation}
\end{center}
\end{frame}

\begin{frame}[t,fragile]{Conclusions}
\begin{itemize}
\item All the strategies have lead to results which are similar to those obtained with user segmentation.
\item Strategy 3 and 4 have shown to be the most effective strategies at approximating real user input (highest rank correlation), with strategy 4 time accuracy profile having the closer visual correspondence with respect to user experiments one. 
\item User evaluation is still the most effective way to evaluate interactive segmentation.
\item Automated evaluation could provide useful and informative results whenever user test cannot be performed (e.g. due to high time consumption).
\end{itemize}
\end{frame}

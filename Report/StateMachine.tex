\section{Controller details} \label{sec:sm}

\begin{center}
\begin{tikzpicture}[shorten >=1pt,node distance=5cm,on grid,auto] 
   \node[state,initial] (EN)   {Exit Nest}; 
   \node[state,thick] (EX) [below=of EN] {Explorer};
   \node[state,thick] (EC) [right=of EN] {Chain Extension};
   \node[state,accepting,thick] (SR) [right=of EC] {Spot Reached};
   \node[state,accepting,thick] (CE) [below =of EC] {Chain End};
   \node[state,accepting,thick] (CM) [right=of CE] {Chain Member};
   \node[state,accepting,thick] (CJ) [below left=of CM] {Chain Junction};
    \path[->] 
    (EN) edge [bend right]  node [left,text width=2cm]  {Out of nest $\wedge \text{} n_b = 0$} (EX)
         edge [bend left]  node [above,text width=2cm]  {$n_b >0$} (EC)
    (EX) edge [bend left]  node [above,text width=2cm]  {$n_b >0$} (EC)
         edge [bend right]  node [below,text width=2cm]  {$n_b = 0 \text{} \wedge$ Out of Nest $\wedge \text{} P_{etob}$} (CE)
    (CE) edge  node  {$n_b > 1$} (CM)
    (CM) edge [bend left] node [text width=2cm] {$n_b > 2$} (CJ)
    (EC) edge  node [above,text width=1.5cm]  {$n_b = 0 \text{} \wedge$ In Nest} (EN)
           edge  node [text width=1.5cm]  {$n_b > 1 \text{} \wedge$ Out of Nest} (EX) 
           edge  node [right,text width=1.5cm] {$n_b > 0 \text{} \wedge$ Out of Nest $\wedge \text{} d_{cb} > d_c$} (CE)
           edge  node {On spot} (SR);
\end{tikzpicture}
\captionof{figure}{Implemented FSM for the \emph{s-bot} chain formation behavior}
\end{center}

\paragraph{Controller parameters}
\begin{center}
\begin{tabular}{|c|c|c|}
\hline
\textbf{Parameter} & \textbf{Description} & \textbf{Value} \\ \hline
$v_{wheel}$ & Wheel velocity (straight movement) & $10 [\frac{cm}{s}]$ \\ \hline
$P_{etob}$ & Probability of starting a chain while being an explorer & 0.05 \\ \hline
$d_{chain}$ & Minimum distance among elements in the chain  & 130 [cm] \\ \hline
$\omega_{ds}$ & Angular rotation velocity of the distance sensor & $2\pi [\frac{rad}{s}]$ \\ \hline
$t_{ns}$ & Delay time on the decision to stop outside the nest & 100 [steps] \\ \hline
\end{tabular}
\captionof{table}{Controller parameters overview}
\label{tab:parameters}
\end{center}

\paragraph{Vector fields}
The resulting forces from the aforementioned vector field are determined from the corresponding sensor readings.
That is, each reading is transformed into a vector whose angle and magnitude depends on the value of the reading itself.
All the vector are expressed in polar coordinates, in the form (magnitude,angle).

\subparagraph{Obstacle avoidance}
\begin{equation}
\mathbf{p_i} = (r_{prox_{i}},\theta_i)  
\end{equation}
where $r_{prox_i}$ corresponds to the value of the reading of the $i^{th}$ proximity sensor and $\theta_i$ is the corresponding angle.

\begin{equation}
\mathbf{sp} = \sum_i \mathbf{p_i}  
\mathbf{F_{obs}} = (1.5,\theta_{sp})  
\end{equation}
where $\theta_{sp}$ corresponds to the angle of $\mathbf{sp}$ vector.

\subparagraph{Distance scanner}
\begin{equation}
\mathbf{ds_i} = (1 - \frac{150 - r_{ds_{i}}}{150-4},\theta_i)  
\end{equation}
where $r_{prox_i}$ corresponds to the value of the $i^th$ reading of the distance and $\theta_i$ is the corresponding angle.

\begin{equation}
\mathbf{sds} = \sum_i \mathbf{ds_i}  
\mathbf{F_{ds}} = (1.0,\theta_{sds})  
\end{equation}
where $\theta_{sds}$ corresponds to the angle of $\mathbf{sds}$ vector.

\subparagraph{Range and Bearing}
\begin{equation}
\mathbf{F_{rab}} = (1.0,\theta_{rab})  
\end{equation}
where $\theta_{rab}$ corresponds to the angle of RAB reading.

\paragraph{Exit nest}
The exit nest state is the initial state of the robots, which is maintained as long as it remains in the nest.
Basically, it consist of a wall avoidance behavior, performed using the distance scanner, rotating with angular speed $\omega_{ds}$.
In fact, the distance scanner is locked in a way that the long range sensors point in the same direction as the robot.

The wall avoidance behavior consist in turning clockwise or counterclockwise (depending on the robot) every time a wall is sensed, until the obstacle is not observed anymore.

Moreover, a random turn is performed if the robot has gone straight for more than 90 steps.
This is done to avoid the fact that the robot could ignore the opening in the nest and keep advancing until the opposite wall is sensed.

If at least one beacon is sensed, the robot heads towards him, in order to be attracted outside the nest.
Otherwise, if it exists without sensing any beacons, it passes in \emph{Explorer} state.
\paragraph{Explorer}
In the \emph{Explorer} state, if no beacon is sensed, the robot moves outsides the nest without making any decision for $t_{ns}$ time steps, then deciding, at each simulation step, whether to stop or not.

The decision is stochastic and occurs with probability $P_{etob}$.

Otherwise, the robot directs himself to the closest sensed beacon.
In any case, if the agent goes back to the nest, it enters the \emph{Exit Nest} state.
\paragraph{Directing to beacon}

\paragraph{Chain End}
The \emph{Chain End} state is reached when a robot connects himself to the current end of the chain, becoming thus the new termination.
In this state, the robot stands still and broadcast information concerning its state, id and chain\_id to the neighboring robots.
The transition to the \emph{Chain Member state} occurs anytime another agent connects to the current one.

\paragraph{Chain Member}
A \emph{Chain Member} only acts as a beacon, broadcasting state information and acting as a relay for the back-propagation of the "Target found" message.

\paragraph{Chain Junction}
The sole purpose of a \emph{Chain Junction} is to offer a branching point in the chain to the robots that are navigating through the chain.
